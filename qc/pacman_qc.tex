\documentclass[12pt]{article}
\usepackage{lmodern}
\usepackage[T1]{fontenc}
\usepackage[dvips,letterpaper,margin=0.75in,bottom=0.75in]{geometry}
\usepackage{cancel}
\usepackage{graphicx}
\usepackage{braket}
\usepackage{latexsym,amssymb,amsmath}
\usepackage{pdfpages}
\usepackage{xcolor}
\usepackage{capt-of}
\usepackage{amsmath}
\usepackage{cite}
\newcommand{\tcr}{\textcolor{red}}
\newcommand{\tcb}{\textcolor{blue}}


\usepackage[american,fulldiode]{circuitikz}
\tikzset{component/.style={draw,thick,circle,fill=white,minimum size =0.75cm,inner sep=0pt}}

\begin{document}
\ctikzset{bipoles/thickness=1}
\ctikzset{bipoles/length=.6cm}


\title{Quality Control for the PACMAN Card}

\maketitle

\section{Overview}

The PACMAN card quality control regimen relies on several different techniques:
\begin{itemize}

\item {\bf Loopback:} Loopback is generally the most effective method
  of feature verification.  The loopback mindset first drives the
  design toward adequate feedback and test input mechanisms, and then
  tests both the feature and its corresponding feedback or test input
  mechanism simultaneously.

\item {\bf Accurate Blame Assessment:} Designing for blame assessment
  forces a modular design that lends itself to effective QC.  For an
  important example, loopback can be performed in the hardware, or in
  firmware (just prior to leaving/ entering the device), or in the
  CPU, so that the source of a failure can immediately be localized to
  hardware, programmable logic, or software.

\item {\bf Design Review:} The PACMAN design team includes a lead
  engineer, and a consulting engineer, and a principile investigator.
  All three understand the board down to the last net list.  The
  PACMAN design is also periodically subjected to independent review.
  
\item {\bf Prototype Systems:} The most effective form of quality
  control is operation in a working system, and the PACMAN card has
  been used continuously in prototype systems since the first version
  in 2020.

\end{itemize}

\section{Feature Verification Overview}

Here we list design features of the PACMAN card
that are tested by a loopback mechanism.  Specific details on
performing the relevant tests are discussed in the hardware checkout
section.
\begin{itemize}
\item The provision of tile voltages VDDD and VDDA is verified by
  onboard current and voltage monitoring ADCs.

\item The POSI and PISO (Digital I/O to Tiles) signals are tested via
  loopback using both test patterns and pseudorandom numbers.

\item The CLK, RESETN/SYNC, and TRIG signals (Digital I/O to Tiles)
  are tested by loopback to a PISO input.

\item The Analog Monitor signals and MUX are tested by connecting the
  positive and negative signal at the ASIC end, then configuring the
  MUX to connect one side to a DAC channel and the other side to the
  onboard high-performance ADC.

\item The onboard high-performance ADC is tested by connecting both
  the positive and negative terminals to independently controlled DAC
  outputs.

\item The front-panel IO is tested via loopback by connecting board
  outputs back in as input using cables plugged into the front-panel.

\item The LEDs are tested with blink patterns.

\end{itemize}

The PACMAN card hosts a linux system, so a broad spectrum of test
software is immediately available to us.  Here we list what we
consider to the be the most important tests of the Linux host.
Specific details on performing the relevant tests are discussed in the
hardware checkout section.

\begin{itemize}
\item Successful system BOOT, visible via UART.
\item Successful remote login via ethernet.
\item Successful transfer of psuedo-random data via ethernet (scp) and the UART terminal (lrzsz).
\item Repeated read and write of psuedo-random numbers to the SD card.
\end{itemize}

Some of the tests described in this section require configuring the
PACMAN card in ways that are not appropriate for a card installed in
the production system.  For this reason, the PACMAN card includes an
expert mode enable two pin jumper.  When a PACMAN card is installed in
the production system, this jumper is removed, and the software
refuses to provide inappropriate test configurations.

\section{Requirement Verification}

In addition to feature operation, the PACMAN has several performance requirements which require quantitative verifications:

\begin{itemize}
 \item The provision of voltage and power across the required ranges
   is tested by applying the appropriate load resistance VDDD and VDDA.
 \item The common mode rejection feature of the DC voltage supply is
   tested by injecting larger than expected common mode noise at the
   input and measuring the provided DC voltage.
 \item The bit error rate is measured during loopback testing of the digital I/O systems.  
 \item The high-frequency noise rejection is tested by injecting a
   high-frequency signal at the input and measuring the signals after
   filtering. (Note: these are the most challenging tests proposed)
\end{itemize}

\section{Hardware Checkout}

This is under development.

Recorded currents for bare board:
\begin{center}
\begin{tabular}{|l|l|l|l|}
\hline
Voltage & Current    &  Current \\
        & (No Jumper) & (Analog Power Enabled) \\
48      & ...         & ... \\
\end{tabular}
\end{center}

\end{document}

